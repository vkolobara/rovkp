%%%%%%%%%%%%%%%%%%%%%%%%%%%%%%%%%%%%%%%%%%%%%%%%%%%%%%%%%%%%%%%%%%%%%%
% LaTeX Example: Project Report
%
% Source: http://www.howtotex.com
%
% Feel free to distribute this example, but please keep the referral
% to howtotex.com
% Date: March 2011 
% 
%%%%%%%%%%%%%%%%%%%%%%%%%%%%%%%%%%%%%%%%%%%%%%%%%%%%%%%%%%%%%%%%%%%%%%
% How to use writeLaTeX: 
%
% You edit the source code here on the left, and the preview on the
% right shows you the result within a few seconds.
%
% Bookmark this page and share the URL with your co-authors. They can
% edit at the same time!
%
% You can upload figures, bibliographies, custom classes and
% styles using the files menu.
%
% If you're new to LaTeX, the wikibook is a great place to start:
% http://en.wikibooks.org/wiki/LaTeX
%
%%%%%%%%%%%%%%%%%%%%%%%%%%%%%%%%%%%%%%%%%%%%%%%%%%%%%%%%%%%%%%%%%%%%%%
% Edit the title below to update the display in My Documents
%\title{Project Report}
%
%%% Preamble
\documentclass[paper=a4, fontsize=11pt]{scrartcl}
\usepackage[T1]{fontenc}
\usepackage{fourier}
\usepackage[utf8]{inputenc}
\usepackage{amsmath,amsfonts,amsthm} % Math packages
\usepackage[pdftex]{graphicx}	
\usepackage{url}
\usepackage[skip=2pt]{caption} % example skip set to 2pt
\usepackage[croatian]{babel}
\usepackage{listings}
\usepackage{enumitem}

%%% Custom sectioning
\usepackage{sectsty}
\allsectionsfont{\centering \normalfont\scshape}


%%% Custom headers/footers (fancyhdr package)
\usepackage{fancyhdr}
\pagestyle{fancyplain}
\fancyhead{}											% No page header
\fancyfoot[L]{}											% Empty 
\fancyfoot[C]{}											% Empty
\fancyfoot[R]{\thepage}									% Pagenumbering
\renewcommand{\headrulewidth}{0pt}			% Remove header underlines
\renewcommand{\footrulewidth}{0pt}				% Remove footer underlines
\setlength{\headheight}{13.6pt}


%%% Equation and float numbering
\numberwithin{equation}{section}		% Equationnumbering: section.eq#
\numberwithin{figure}{section}			% Figurenumbering: section.fig#
\numberwithin{table}{section}				% Tablenumbering: section.tab#

%%% Maketitle metadata
\newcommand{\horrule}[1]{\rule{\linewidth}{#1}} 	% Horizontal rule

\title{
		%\vspace{-1in} 	
		\usefont{OT1}{bch}{b}{n}
		\normalfont \normalsize \textsc{Fakultet Elektrotehnike i Računarstva} \\ [25pt]
		\horrule{0.5pt} \\[0.4cm]
		\huge 2. Domaća zadaća - ROVKP \\
		\horrule{2pt} \\[0.5cm]
}
\author{
		\normalfont 								\normalsize
        Vinko Kolobara\\[-3pt]		\normalsize
        \today
}
\date{}


%%% Begin document
\begin{document}
\maketitle

\section{Zadatak: Izrada MapReduce programa}
Koliko različitih vozila se nalazi u ulaznoj datoteci?\\
U ulaznoj datoteci se nalazi $9834$ različitih vozila.\\

Koliko je trajala najdulja ukupna vožnja jednog taksija? Koja je minimalna, a koja najdulja vožnja tog taksija?\\
Najdulja ukupna vožnja jednog taksija je trajala 225600s, minimalna 0s, a maksimalna 3360s.\\

Koje ste promjene morali napraviti na izvornom kodu prilikom uvođenja optimizacijske funkcije Combine?\\
Potrebno je samo postaviti sa job.setCombinerClass razred koji će se koristiti za Combiner, i to isti razred koji se koristi za Reducer.\\

Koliko vremena su se izvodile inačice programa? Je li to u skladu s vašim očekivanjem? Objasnite zašto.\\

Inačice programa su se izvodile približno jednako (26915ms bez, 25750ms sa Combinerom), ali ipak sevarijanta s Combiner-om pokazala nešto bržom. Nije bilo za očekivati veću razliku zbog toga što se sve izvodilo u virtualnom stroju u pseudo-raspodijeljenom načinu s malom datotekom itd. pa hadoop nije mogao pokazati svoju moć.

\pagebreak

\section{Zadatak: Korištenje obrasca dijeljenja podataka (partitioning pattern)}
Koliko je vožnji realizirano u pojedinom području, tj. u užem centru i u širem gradskom području, i to tako da je broj putnika bio 1, 2-3 putnika ili 4 i više putnika?\\
U užem centru je realizirano $500061$, a u širem gradskom području $763328$ vožnji.\\

Koje ste promjene morali napraviti na izvornom kodu prilikom uvođenja funkcije Partition?\\
Potrebno je samo poslu postaviti koji će se razred koristiti za Partitioner sa job.setPartitionerClass.\\

Koliko je vožnji navedeno u svakoj podskupini? 
\begin{itemize}

\item Uži centar, 1 putnik - 244814
\item Uži centar, 2-3 putnika - 70458
\item Uži centar, 4+ putnika - 86294
\item Šire područje, 1 putnik - 525525
\item Šire područje, 2-3 putnika - 151509
\item Šire područje, 4+ putnika - 184789

\end{itemize}

\pagebreak

\section{Zadatak: Izvršavanje ulančanih MapReduce programa}
Koliko ste MapReduce poslova izvršili u vašem kôdu? \\
Jedan posao za razdvajanje podataka na 6 particija, i još po jedan posao za svaku particiju što je ukupno 7 MapReduce poslova.\\

Koliko je različitih taksija realiziralo vožnje u pojedinom području, tj. u užem centru i u širem gradskom području,
i to tako da je broj putnika bio 1, 2-3 putnika ili 4 i više putnika?

\begin{itemize}

\item Uži centar, 1 putnik - 6601
\item Uži centar, 2-3 putnika - 4138
\item Uži centar, 4+ putnika - 3425
\item Šire područje, 1 putnik - 8714
\item Šire područje, 2-3 putnika - 5487
\item Šire područje, 4+ putnika - 3983

\end{itemize}

\end{document}