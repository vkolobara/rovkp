%%%%%%%%%%%%%%%%%%%%%%%%%%%%%%%%%%%%%%%%%%%%%%%%%%%%%%%%%%%%%%%%%%%%%%
% LaTeX Example: Project Report
%
% Source: http://www.howtotex.com
%
% Feel free to distribute this example, but please keep the referral
% to howtotex.com
% Date: March 2011 
% 
%%%%%%%%%%%%%%%%%%%%%%%%%%%%%%%%%%%%%%%%%%%%%%%%%%%%%%%%%%%%%%%%%%%%%%
% How to use writeLaTeX: 
%
% You edit the source code here on the left, and the preview on the
% right shows you the result within a few seconds.
%
% Bookmark this page and share the URL with your co-authors. They can
% edit at the same time!
%
% You can upload figures, bibliographies, custom classes and
% styles using the files menu.
%
% If you're new to LaTeX, the wikibook is a great place to start:
% http://en.wikibooks.org/wiki/LaTeX
%
%%%%%%%%%%%%%%%%%%%%%%%%%%%%%%%%%%%%%%%%%%%%%%%%%%%%%%%%%%%%%%%%%%%%%%
% Edit the title below to update the display in My Documents
%\title{Project Report}
%
%%% Preamble
\documentclass[paper=a4, fontsize=11pt]{scrartcl}
\usepackage[T1]{fontenc}
\usepackage[utf8]{inputenc}
\usepackage{amsmath,amsfonts,amsthm} % Math packages
\usepackage[pdftex]{graphicx}	
\usepackage{url}
\usepackage[skip=2pt]{caption} % example skip set to 2pt
\usepackage[croatian]{babel}
\usepackage{listings}
\usepackage{enumitem}

%%% Custom sectioning
\usepackage{sectsty}
\allsectionsfont{\centering \normalfont\scshape}


%%% Custom headers/footers (fancyhdr package)
\usepackage{fancyhdr}
\pagestyle{fancyplain}
\fancyhead{}											% No page header
\fancyfoot[L]{}											% Empty 
\fancyfoot[C]{}											% Empty
\fancyfoot[R]{\thepage}									% Pagenumbering
\renewcommand{\headrulewidth}{0pt}			% Remove header underlines
\renewcommand{\footrulewidth}{0pt}				% Remove footer underlines
\setlength{\headheight}{13.6pt}


%%% Equation and float numbering
\numberwithin{equation}{section}		% Equationnumbering: section.eq#
\numberwithin{figure}{section}			% Figurenumbering: section.fig#
\numberwithin{table}{section}				% Tablenumbering: section.tab#

%%% Maketitle metadata
\newcommand{\horrule}[1]{\rule{\linewidth}{#1}} 	% Horizontal rule

\title{
		%\vspace{-1in} 	
		\usefont{OT1}{bch}{b}{n}
		\normalfont \normalsize \textsc{Fakultet Elektrotehnike i Računarstva} \\ [25pt]
		\horrule{0.5pt} \\[0.4cm]
		\huge 3. Domaća zadaća - ROVKP \\
		\horrule{2pt} \\[0.5cm]
}
\author{
		\normalfont 								\normalsize
        Vinko Kolobara\\[-3pt]		\normalsize
        \today
}
\date{}


%%% Begin document
\begin{document}
\maketitle

\section{Zadatak: Indeksiranje tekstualne kolekcije}
Koliko se zapisa nalazi u izlaznoj datoteci?\\
U izlaznoj datoteci se nalazi $11175$ $100 \choose 2$ zapisa.\\

Koja šala je najsličnija šali s ID-jem 1?\\
Najsličnija je ona šala sa ID-jem 87.\\

Vidite li zašto su te dvije šale slične?\\
U obje šale se spominje doktor.\\

Što mislite, hoće li preporuka po sadržaju imati smisla u slučaju ovih šala?\\
Trebala bi imati smisla. Ljudima bi se trebale sviđati šale slične po sadržaju\\


\pagebreak

\section{Zadatak: Izgradnja i evaluacija centraliziranog preporučitelja}
Kojih 10 preporuka je za korisnika s ID-jem 220 je izračunao prvi, a koje drugi preporučitelj?\\
Prvi preporučitelj: 141, 140, 44, 22, 36, 52, 86, 37, 129, 43\\
Drugi preporučitelj: 62, 68, 105, 66, 53, 104, 35, 114, 148, 106\\

Koje preporučitelj ima bolju kvalitetu?\\
Nešto bolju kvalitetu ima prvi preporučitelj.\\

Je li za ove ulazne podatke bolje koristiti mjeru log-likelihood ili Pearsonovu korelaciju u slučaju drugog preporučitelja?\\
Bolje je koristiti log-likelihood.

\pagebreak

\section{Zadatak: Pokretanje raspodijeljenog preporučitelja}
\begin{lstlisting}[language=bash,caption={Pokretanje raspodijeljenog preporučitelja}]
mahout recommenditembased  \
--similarityClassname SIMILARITY_PEARSON_CORRELATION \
--input data/jester_ratings.csv \
--usersFile data/users.txt \
--output data/result \
--numRecommendations 10
\end{lstlisting}



Koliko ste MapReduce poslova izvršili u vašem kôdu? \\
9 MapReduce poslova je izvršeno.\\

Pogledajte izlazne datoteke i objasnite postoji li razlika u izračunatim preporukama u odnosu na
preporučitelja iz 2. zadatka?\\
Postoje razlike, prvenstveno zato što je u 2. zadatku korištena druga mjera sličnosti koja ne daje iste rezultate kao Pearsonova korelacija.

\end{document}