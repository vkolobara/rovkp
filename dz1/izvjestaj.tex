%%%%%%%%%%%%%%%%%%%%%%%%%%%%%%%%%%%%%%%%%%%%%%%%%%%%%%%%%%%%%%%%%%%%%%
% LaTeX Example: Project Report
%
% Source: http://www.howtotex.com
%
% Feel free to distribute this example, but please keep the referral
% to howtotex.com
% Date: March 2011 
% 
%%%%%%%%%%%%%%%%%%%%%%%%%%%%%%%%%%%%%%%%%%%%%%%%%%%%%%%%%%%%%%%%%%%%%%
% How to use writeLaTeX: 
%
% You edit the source code here on the left, and the preview on the
% right shows you the result within a few seconds.
%
% Bookmark this page and share the URL with your co-authors. They can
% edit at the same time!
%
% You can upload figures, bibliographies, custom classes and
% styles using the files menu.
%
% If you're new to LaTeX, the wikibook is a great place to start:
% http://en.wikibooks.org/wiki/LaTeX
%
%%%%%%%%%%%%%%%%%%%%%%%%%%%%%%%%%%%%%%%%%%%%%%%%%%%%%%%%%%%%%%%%%%%%%%
% Edit the title below to update the display in My Documents
%\title{Project Report}
%
%%% Preamble
\documentclass[paper=a4, fontsize=11pt]{scrartcl}
\usepackage[T1]{fontenc}
\usepackage{fourier}
\usepackage[utf8]{inputenc}
\usepackage{amsmath,amsfonts,amsthm} % Math packages
\usepackage[pdftex]{graphicx}	
\usepackage{url}
\usepackage[skip=2pt]{caption} % example skip set to 2pt
\usepackage[croatian]{babel}
\usepackage{listings}
\usepackage{enumitem}

%%% Custom sectioning
\usepackage{sectsty}
\allsectionsfont{\centering \normalfont\scshape}


%%% Custom headers/footers (fancyhdr package)
\usepackage{fancyhdr}
\pagestyle{fancyplain}
\fancyhead{}											% No page header
\fancyfoot[L]{}											% Empty 
\fancyfoot[C]{}											% Empty
\fancyfoot[R]{\thepage}									% Pagenumbering
\renewcommand{\headrulewidth}{0pt}			% Remove header underlines
\renewcommand{\footrulewidth}{0pt}				% Remove footer underlines
\setlength{\headheight}{13.6pt}


%%% Equation and float numbering
\numberwithin{equation}{section}		% Equationnumbering: section.eq#
\numberwithin{figure}{section}			% Figurenumbering: section.fig#
\numberwithin{table}{section}				% Tablenumbering: section.tab#

%%% Maketitle metadata
\newcommand{\horrule}[1]{\rule{\linewidth}{#1}} 	% Horizontal rule

\title{
		%\vspace{-1in} 	
		\usefont{OT1}{bch}{b}{n}
		\normalfont \normalsize \textsc{Fakultet Elektrotehnike i Računarstva} \\ [25pt]
		\horrule{0.5pt} \\[0.4cm]
		\huge 1. Domaća zadaća - ROVKP \\
		\horrule{2pt} \\[0.5cm]
}
\author{
		\normalfont 								\normalsize
        Vinko Kolobara\\[-3pt]		\normalsize
        \today
}
\date{}


%%% Begin document
\begin{document}
\maketitle

\section{Zadatak}

\begin{enumerate}
\setcounter{enumi}{1}


\item{
\begin{lstlisting}[language=bash]
  $ start-dfs.sh
\end{lstlisting}
}

\item{
\begin{lstlisting}[language=bash]
  $ hdfs dfs -ls /user/rovkp
\end{lstlisting}
}

\item{
\begin{lstlisting}[language=bash]
  $ wget http://svn.tel.fer.hr/gutenberg.zip
\end{lstlisting}
}

\item{
\begin{lstlisting}[language=bash]
  $ hdfs dfs -copyFromLocal gutenberg.zip /user/rovkp
\end{lstlisting}
}

\item{
\begin{lstlisting}[language=bash]
  $ hdfs fsck /user/rovkp/gutenberg.zip
\end{lstlisting}

\begin{enumerate}[label=\alph*.]

\item {
datoteka se sastoji od 2 bloka
}

\item {
replikacijski faktor je 1
}

\item {
hdfs je napravljen da jako dobro radi sa velikim datotekama, što datoteka od 151MB nije
}

\end{enumerate}
}

\item{
\begin{lstlisting}[language=bash]
  $ mv gutenberg.zip gutenberg_backup.zip
\end{lstlisting}
}

\item{
\begin{lstlisting}[language=bash]
  $ hdfs dfs -copyToLocal /user/rovkp/gutenberg.zip .
\end{lstlisting}
}

\item{
\begin{lstlisting}[language=bash]
  $ echo `md5sum gutenberg.zip` | \ 
   awk '{print $1 " gutenberg_backup.zip";}' | \ 
   md5sum -c
\end{lstlisting}
}


\end{enumerate}

\section{ZADATAK}

\lstinputlisting[language=Java, breaklines=true, tabsize=2]{java/src/main/java/hr/vinko/rovkp/dz1/zad2/GutenbergToTxtLocal.java}

\paragraph{}
Konačna veličina datoteke gutenberg\_books.txt je 418.242.611 byte-ova.\\

Ukupno je pročitano $8481553$ redaka.\\

Za datoteku su potrebna 4 bloka, a uz faktor replikacije 3 na HDFS-u bi se stvorilo 12 blokova. \\

Program se izvodio 16879ms, a očekivano vrijeme na HDFS-u je bar četiri puta brže.

\pagebreak
\section{ZADATAK}

\lstinputlisting[language=Java, breaklines=true, tabsize=2]{java/src/main/java/hr/vinko/rovkp/dz1/zad3/HadoopJavaTest.java}

\end{document}